
%%******************************************************************************
%% SECTION -•	Métodos 
Describe the steps you completed during your investigation. This is your procedure. Be sufficiently detailed that anyone could read this section and duplicate your experiment. Write it as if you were giving direction for someone else to do the lab. It may be helpful to provide a Figure to diagram your experimental setup.

%%******************************************************************************

\subsection{Métodos}
Esta seção está subdividida em: Logística de materiais e dispositivos,
planejamento do experimento, montagem, execução e colheta de dados com análise,
desmontagem e fechamento.

\subsubsection{Logística de materiais e dispositivos}
Esta subseção abrange tanto pesquisa e escolha de malas/cases para
transporte de equipamentos e ferramentas, quanto organização interna dos
cases, proteção para transporte aéreo, meios de locomoção e dificuldades
encontradas.

A pesquisa e compra do pelican-case para transporte de ferramentas foi
realizada duas semanas antes da viagem, porém a alteração no escopo dos experimentos, na mesma
semana, exigiu uma nova pesquisa. Os professores Ramon e Jacoud avaliaram o
tempo ainda disponível e consideraram a viagem uma boa oportunidade para testar
os novos sensores que foram entregues:
profundímetro da Velki e sensor inercial (IMU), dos projetos LUMA e DORIS. Este
último faria o papel do inclinômetro no escopo do projeto. 

Devido ao teste extra, houve a necessidade de um novo projeto para a eletrônica
embarcada à prova d'água, cabos com emendas submarinas e montagem da estrutura mecânica, além da readaptação da placa eletrônica para os novos sensores com acréscimos de novos
CIs e uma grande reestruturação do software. A compra de novos componentes,
cases e cabos foram no centro da cidade, Rua República do Líbano, pelo método de
reembolso e com transporte particular.

O umbilical proposto para o novo teste da eletrônica é composto por 12 vias com
40m de comprimento, sem carretel. Os outros diversos cabos da eletrônica, fonte
e baterias exigiram a aquisição de um case $65x65x65cm$ com rodas, totalizando um
peso de $70kg$. Não foi possível realizar a organização interna do case, já que
o tempo da última semana foi reservado, em sua maior parte, para a
reestruturação eletrônica e de software a fim de garantir o último teste e a
obtenção de dados. 

O transporte dos cases foram realizados do laboratório à Usina nas
seguintes etapas:
\begin{itemize}
  \item Laboratório-Aeroporto: carro particular;
  \item Aeroporto: despachados;
  \item Aeroporto-Nova Mutum Paraná: carro alugado Hilux;
  \item Nova Mutum Paraná-Usina: carro alugado Hilux;
\end{itemize} 



 

\label{metodos}


