
%%******************************************************************************
%% SECTION - Purpose 

\subsection{Propósito}
A eletrônica embarcada pode ser subdividida em um projeto mecânico e um projeto
eletrônico. O projeto mecânico deve garantir resistência moderada à choque,
vibração e pressão de 30m d'água. O projeto eletrônico é composto pelos
dispositivos: profundímetro, sensor inercial (IMU), gateway Ethernet, e
placa eletrônica customizada, cabos internos, umbilical, conectores
submarinos, baterias e botão liga/desliga.

Como o projeto mecânico se assemelha ao que foi utilizado na primeira versão do
projeto LUMA, ROV submarino com expedição Antártida, os testes da eletrônica embarcada tiveram
como foco o projeto da eletrônica:
\begin{itemize}
  \item Alimentação dos dispositivos a partir da placa customizada e
  baterias;
  \item Componentes da placa customizada;
  \item Botão liga/desliga;
  \item Comunicação entre placa customizada e base;
  \item Dispositivos integrados à eletrônica: Sonar, IMU,
  Profundímetro e sensores indutivos;
  \item Teste final da eletrônica embarcada, após organização dentro do tubo; 
\end{itemize} 
\label{proposito_sonar}

