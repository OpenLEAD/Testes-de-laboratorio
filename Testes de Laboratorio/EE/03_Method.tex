
%%******************************************************************************
%% SECTION -•	Métodos 
%Describe the steps you completed during your investigation. This is your
% procedure. Be sufficiently detailed that anyone could read this section and duplicate your experiment. Write it as if you were giving direction for someone else to do the lab. It may be helpful to provide a Figure to diagram your experimental setup.

%%******************************************************************************

\subsection{Métodos}
Foram realizados dois tipos de testes com o sensor indutivo: teste de
operação e teste de comunicação via LAN.

\subsubsection{Teste de operação}
Em meados de março 2014, foram realizados testes básicos de funcionamento do
sensor indutivo. O teste consistiu em ligar o sensor em uma fonte de $15V$ e
observar o comportamento quando metais são aproximados. Foram testadas variações
de tensão entre $12V$ e $21V$, imersão do dispositivo em água e alcance do
sensoriamento aproximando metais pela lateral e pela direção alvo.

Na montagem do experimento, o sensor indutivo foi conectado ao cabo à prova
d'água M12, também da Pepperl-Fuchs. O rabicho, outra ponta do cabo, são quatro
vias com as cores marrom (alimentação do sensor), azul (GND), preta (sinal) e
branca (não conectada). O sensor indutivo apresenta LEDs que indicam o
funcionamento do dispositivo e a presença de objetos metálicos.

\subsubsection{Teste de comunicação via LAN}
O objetivo do sensor indutivo é verificar a presença de stoplogs, e com esta
análise podermos inferir se houve o contato com a garra pescadora. Dessa forma,
os dados dos sensores indutivos deverão ser enviados ao operador por rede LAN.
A saída do sensor passará por um divisor de tensão para ser regulada a 3.3V e
daí será entrada como porta GPIO de um gateway GPIO/Ethernet, que utiliza um
microcontrolador PIC. O esquema eletrônico pode ser verificado na
figura~\ref{fig:indu_banc}.

\begin{figure}[h!]
 \centering
 \includegraphics[width=1\columnwidth]{indutivo/figs/indutivo_bancada.png}
 \caption{Eletrônica do sensor indutivo}
 \label{fig:indu_banc}
 \end{figure}


 

\label{metodos}


